% !TEX root = ../main.tex

\appendix
\section{\\Sample Code}
This section contains the pseudocode for the circuits used in both the proof of liabilities, and proof of inclusion.

\subsection{Proof of liabilities}
\label{subsec:plc}
\begin{lstlisting}[language=Python, caption=Liabilities circuit pseudocode]
# Define a function for the Merkle tree sum circuit
def sum_merkle_tree(levels, inputs, balances[inputs], 
                        email_hashes[inputs]):
    # Outputs variables for sum, root hash, and range checks
    root_sum = 0
    root_hash = 0
    not_negative = False
    all_small_range = False

    # Define signals lists for sum and hash nodes at each level
    sum_nodes = [[0] * inputs for _ in range(levels+1)]
    hash_nodes = [[0] * inputs for _ in range(levels+1)]

    # Initialize variables
    level_size = inputs
    max_bits = 100
    temp_not_big = 0
    temp_not_negative = 0

    # Loop through each input
    for i in range(inputs):
        # Assign input values to hash and sum nodes
        hash_nodes[0][i] = email_hashes[i]
        sum_nodes[0][i] = balances[i]

        # Perform range check and negative check
        rangecheck = RangeCheck(maxBits,balances[i])
        tempNotBig += rangecheck
        negativecheck = NegativeCheck(newValues[i])
        tempNotNegative += negativecheck

    # Check if all balances are within a small range
    if temp_not_big == inputs:
        all_small_range = True

    # Check if all balances are non-negative
    if temp_not_negative == inputs:
        not_negative = True

    # Loop through each level
    for i in range(levels):
        # Loop through each pair of nodes at the current level
        for j in range(0, level_size, 2):
            # Store sum and root hash for the next level
            sum_nodes[i+1][nextLevelSize] = 
            sum_nodes[i][j] + sum_nodes[i][j+1]
            hash_nodes[i+1][nextLevelSize] = 
            hash_nodes[i][j] + hash_nodes[i][j+1]

        # Update the size of the current level
        levelSize = nextLevelSize;
        nextLevelSize = 0;

    # Assign final sum and root hash values
    root_sum = sum_nodes[levels][0]
    root_hash = hash_nodes[levels][0]

    return root_sum, root_hash, not_negative, all_small_range

\end{lstlisting}

\subsection{Proof of inclusion}
\label{subsec:pic}
\begin{lstlisting}[language=Python, caption=Inclusion circuit pseudocode]
# Define a function for the inclusion proof circuit
def inclusion(levels, neighborsSum, neighborsHash, neighborsBinary, 
                step_in, sum, rootHash, userBalance, userEmailHash):
    # Initialize sum and hash nodes
    sumNodes = [0] * (levels+1)
    hashNodes = [0] * (levels+1)
    sumNodes[0] = userBalance
    hashNodes[0] = userEmailHash

    # Iterate through each level
    for i in range(levels):
        # Update hash node
        if neighborsBinary[i]:
            hashNodes[i+1]=getHash(neighborsHash[i],
            neighborsSum[i],hashNodes[i],sumNode[i])
        else:
            hashNodes[i+1]=getHash(hashNodes[i],sumNode[i],
            neighborsHash[i],neighborsSum[i])
        # Update sum node
        sumNodes[i + 1] = neighborsSum[i] + sumNodes[i]

    # Check validity of root hash
    validHash = IsEqual([hashNodes[levels], rootHash])

    # Check validity of sum
    validSum = IsEqual([sumNodes[levels], sum])

    return validSum,validHash
\end{lstlisting}

\subsection{Change circuit}
\label{subsec:plcc}
\begin{lstlisting}[language=Python, caption=Liabilities change circuit pseudocode]
# Define a function for the liabilities change proof circuit
def liabilities(levels, changes,oldEmailHash[changes],
        oldValues[changes],newEmailHash[changes],newValues[changes],
        tempHash[changes+1],tempSum[changes+1],
        neighborsSum[changes][levels],neighborsHash[changes][levels],
        neighborsBinary[changes][levels]):         
    # Calculate newRootHash and newSum
    newRootHash = tempHash[changes+1]
    newSum = tempSum[changes+1]
    oldSum = tempSum[0]
    oldRootHash = tempHash[0]
    currentSum = oldSum

    # Part 1: Check validity of new values
    tempNotBig = 0
    tempNotNegative = 0
    maxBits = 100

    # Iterate through each change
    for i in range(changes):
        # Calculate currentSum
        currentSum += newValues[i] - oldValues[i]

        # Perform range check and negative check
        rangecheck = RangeCheck(maxBits,newValues[i])
        tempNotBig += rangecheck
        negativecheck = NegativeCheck(newValues[i])
        tempNotNegative += negativecheck

    # Check if all new values are within range
    allSmallRange = IsEqual([changes, tempNotBig])

    # Check if all new values are not negative
    notNegative = IsEqual([changes, tempNotNegative])

    # Check if newSum equals currentSum
    equalSum = IsEqual([newSum, currentSum])

    # Part 2: Check validity of old and new paths
    # Ensure that old root + change = temp root
    tempValidHash = [0] * (changes + 1)
    tempValidSum = [0] * (changes + 1)
    tempOldHashEqual = [0] * (changes + 1)
    tempOldSumEqual = [0] * (changes + 1)
    tempValidHash[0] = 1
    tempValidSum[0] = 1

    #For each level, we are veryfing the inclusion of the value changing
    #and the new value.
    #hashNode[0] and sumNodes[0] is for the inclusion of the value changing
    #hashNode[1] and sumNodes[1] is for the inclusion of the new value

    for j in range(changes):
        for i in range(levels):
        if neighborsBinary[j][i]:
            hashNodes[0][j][i+1]=getHash(neighborsHash[j][i],
            neighborsSum[j][i],hashNodes[0][j][i],sumNode[0][j][i])
            hashNodes[1][j][i+1]=getHash(neighborsHash[j][i],
            neighborsSum[j][i],hashNodes[1][j][i],sumNode[1][j][i])
        else:
            hashNodes[0][j][i+1]=getHash(hashNodes[0][j][i],
            sumNode[0][j][i],neighborsHash[j][i],neighborsSum[j][i])
            hashNodes[1][j][i+1]=getHash(hashNodes[1][j][i],
            sumNode[1][j][i],neighborsHash[j][i],neighborsSum[j][i])

        sumNodes[0][j][i+1] = sumNode[0][j][i] + neighborsSum[j][i]
        sumNodes[1][j][i+1] = sumNode[0][j][i] + neighborsSum[j][i]
            

        # Old calculated old hash is in tempHash
        hashEqual = IsEqual(hashNodes[0][j][levels], tempHash[j])
        tempOldHashEqual[j+1] = tempOldHashEqual[j] * hashEqual

        # New temp hash is valid
        hashEqual = IsEqual(hashNodes[1][j][levels], tempHash[j+1])
        tempValidHash[j+1] = tempValidHash[j] * hashEqual

        # Old calculated sum is in tempSum
        sumEqual = IsEqual(sumNodes[0][j][levels], tempSum[j])
        tempOldSumEqual[j+1] = tempOldSumEqual[j] * sumEqual
        
        # New temp sum is valid
        sumEqual = IsEqual(sumNodes[1][j][levels], tempSum[j+1])
        tempValidSum[j+1] = tempValidSum[j] * sumEqual

    validHash = tempValidHash[changes] * tempOldHashEqual[changes]
    validSum = tempValidSum[changes] * tempOldSumEqual[changes]

    return (notNegative,allSmallRange,validHash, 
            validSum,newRootHash,newSum)
    
\end{lstlisting}