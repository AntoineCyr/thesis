% !TEX root = ../main.tex

\chapter{Proof construction}
We will be using Circom to generate an airthmetic circuit, and SnarkJS for the proof generation.

\paragraph{Circom}
Circom is a domain-specific language for creating arithmetic circuits, which are used in zk-SNARKs.
The circuit code can be written to specify the desired constraints.
Circom allows expressing the circuit's arithmetic operations, constraints, input and output in a concise and readable manner.
Once the circuit is designed, it needs to be compiled into a format suitable for zk-SNARKs.


\paragraph{SnarkJS}
SnarkJS is a JavaScript library that provides tools for working with zk-SNARKs, including circuit compilation.
After compilation, SnarkJS facilitates generating zk-SNARK proofs for specific instances of the circuit.
SnarkJS also provides utilities for verifying the proofs.

\section{Proof of liabilities}
The proof of liabilities operates on a list of balances and a list of email hashes as private inputs.
The first purpose of the circuit is to validate that all values are non-negative and that all balances fall within a specified range. 
These verifications are crucial to prevent overflow or underflow issues, given that the operations occur within a finite field. 

Subsequently, the proof of liabilities constructs a Merkle tree and provides outputs the total balance sum and the root hash of the Merkle tree.

Inputs:
\begin{enumerate}

    \item List of balance (private)
    
    \item List of email hash (private)
    
    \end{enumerate}

Outputs:
\begin{enumerate}
    Balance Sum (public)
    Root hash (public)
    No negative values (private) - boolean
    All small range (private) - boolean
    \end{enumerate}

This proof of liabilities operates as intended because it returns the sum of the liabilities, which is exact because of the checks. 
It also returns the root hash, insuring you cannot alter any values insiside the merkle tree. The merkle tree is hidden so that we do not
give any information about users and their balances.
The root hash will be used to verify the inclusion of the balances.


In a complete proof of reserves, the balance sum would be a private output. We would have another circuit proving that the sum of liabilities is smaller
than the sum of assets, without revelaing the balance.


\section{Proof of inclusion} 


\section{Daily proof of liabilities} 

\section{Daily proof of inclusion} 





%Mina proof of recursion